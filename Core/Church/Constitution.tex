%% LyX 2.0.6 created this file.  For more info, see http://www.lyx.org/.
%% Do not edit unless you really know what you are doing.
\documentclass[english]{article}
\usepackage[T1]{fontenc}
\usepackage[latin9]{inputenc}
\usepackage{geometry}
\geometry{verbose,tmargin=3cm,bmargin=3cm}
\usepackage{babel}
\begin{document}

\title{Church Constitution}


\author{David Guenther}

\maketitle

\section{Purpose}

The church is a community for a group of believers who live in close
proximity to each other. The ultimate goal of the church is provide
the framework necessary for meaningful discipleship to take place.
The church collectively seeks to grow individuals spiritually though
a connection to God (the Word being a key component), provide emotional/mental
help to those who need it, and physical help to those who need it
as well.

The Church is the representation of Christ on earth. While the church
has often gone astray from her Lord and Maker, she nevertheless remains
loved by God. We hold a commitment to the church, but it is even more
then that, for we are the church. We hold a commitment to ourselves
to be in community with other believers. We believe in being in community
with other believers, and we make it our focus to continue that.

In all of our efforts for spiritual growth, the church must be involved.
We cannot raise up students who love God, if they have no learned
to love God, if they have not learned to love the Bride of Christ.
We cannot raise up good kids in our homes, if they have not learned
to love God with all of their heart. We may discipline them, but if
they do not have Christ, we have given them nothing.

Everything hangs on the church, for it is responsible for keeping
the faith.

The Rock of the Church is her husband, Christ Jesus our Lord. As such,
\begin{itemize}
\item From Him:

\begin{itemize}
\item What He says {[}Word of God{]} (listening to and responding to what
God has said)
\item What He has given us

\begin{itemize}
\item {[}Fruit of the Spirit{]} (love, joy, peace, etc, lived out in each
member)
\item {[}Gifts of the Spirit{]} (teaching, evangelism, encouragement...etc)
\item His Son, and through Him, right relationship with God
\end{itemize}
\end{itemize}
\item Our Response 

\begin{itemize}
\item Our remembrance of Him {[}communion{]} (eating together to commemorate
what Christ has done)
\item Our public declaration of our allegiance to Him {[}Baptism{]}
\item Our Relationship to Him

\begin{itemize}
\item Our Obedience to Him
\item Our Worship of Him
\item Our Prayer to Him
\end{itemize}
\end{itemize}
\end{itemize}
If any of these aspects are ignored, the church suffers


\section{Roles}


\subsection{Apostles}

Apostles serve to establish new churches, but going out and proclaiming
the truth where no one has yet heard the Gospel. An Apostle views
these new churches as new-born children, continuing to care for them
until they reach a maturity in the faith.


\subsection{Prophets}

Prophets serve to reveal the thoughts of God to the church. Their
words must be tested by other prophets, as a safe-guard against heresy.


\subsection{Evangelists}

Evangelists serve to bring the Gospel to people in the same location
as their church. Once a person has committed their life to the Lord,
the evangelist can provide the necessary connection to an church to
help this new believer continue to grow as a Christian.


\subsection{Elders}

Elders serve to guide and protect the people who belong to the church.
As this is a role which has much influence over others, strict guidelines,
as outlined in the NT, must be followed without compromise (1 Tim
3, Titus 1).


\subsection{Teachers}

Teachers serve to accurately teach the Word of God.


\subsection{Deacons}

Deacons serve to serve, helping the church practically so that others
can focus on their ministries.


\subsection{Priests}

Everyone is considered a priest, and as such, have access to God.
No one has exclusive access to God, and cannot specific y how another
comes to God. Also, each individual has something to contribute to
the body of Christ. (1 Peter 2:9, 1 Cor 12:12)


\section{Meetings}


\subsection{Core Groups}

The church meets together in core groups regularly to disciple each
other. These groups ideally range from 3 to 12 people. Meeting informally,
for the purpose of discipleship, fellowship, and life. Core Groups
meet frequently during the week, and thus are best constructed with
local people. Communion is celebrated in this group, as they meet
over meals, sing songs, and discuss their day.


\subsection{Mantle Groups}

The church meets together in mantle groups regularly to encourage
each other. These groups consist of no more then 72 people, consisting
of 6 or more core groups, composed of elders, teachers, prophets,
encourager's, songs, etc. (1 Cor 11:33, 12:26, Hebrews 3:13, 10:24,
25). It is in this group that communion, baptism, encouragement, teaching,
prophecies, and praying happens. Mantle Groups meet at a minimum once
a week, immediately following the crust group.


\subsection{Crust Group}

The church meets together in crust groups to keep the unity of the
church. This group consists of up to 12 groups of 72 people (864).
It is in this group that singing corporately happens, teaching from
the various teachers dispersed throughout the mantle groups, prophecies,
and missions/ministry focus. Crust groups meet at a minimum once a
month. 


\subsection{Global Groups}

Once a year the church meets, gathering the believers from the surrounding
city, nation, and world, to encourage each other, bring others up
to date about what is happening, break bread, and worship together.


\section{Serving}


\subsection{Ministries}

The church participates in various ministries, from serving the church,
to serving the community, being equipped by the APEET; doing the work
laid out before them. (Ephesians 2:10, 4:11-12, 1 Corinthians 12:27-31,
1 Peter 4:10). Ministries range across all groups, allowing those
called to participate.


\subsection{Missions}

The church also sends out apostles to create new assemblies in other
locations. Also giving help to other assemblies that may be struggling.
(Romans 10:15, 1 Corinthians 16:1-4, John 17:20-23, Proverbs 27:10).


\section{Procedures}


\subsection{Called to Ministry, Missions}

Those who feel called, or others see as serving in a certain role,
will be first mentioned in the core groups. Once the core group affirms
the direction informally, it is brought to the Mantle group, which
votes unanimously on the person in the new role. If someone does not
vote yes, it is then an opportunity to provide constructive criticism
on where they could serve, or steps needed before they can serve.
Once this decision has been made, the person is commissioned at the
crust group, having hands laid upon them by the elders, and then prayed
for by the entire assembly. All members in ministry are considered
deacons, all members in missions are considered apostles.


\subsection{Updates}

For those serving in various capacities, a monthly report must be
given, and if not in person, read to the entire crust assembly.


\subsection{Stepping down}

If the a member serving feels this is no longer a direction for them,
they will mention this in their monthly report. At this point, the
congregation will pray for someone to rise up to fill their place.
This replacing will follow the normal procedure.


\subsection{Growth}

The church must be called to divide at a certain point, to maintain
the internal integrity. If a core group reaches 16, two groups of
8 will be formed (keep families together). If a mantle group reaches
80, the group will divide, having half of the core groups forming
a new church, and the other half starting a new church. In all cases,
one must see the transition as a new starting, and not a departing
from the center group. If a mantle crust group reaches 880, they group
is split into two group once again. If facilities can no longer house
the Global group, multiple global groups will be started. It is assumed
that a crust group will all attend the same global group (if they
go).

These are guidelines, the numbers are less important. What is important
is multiplication of believers, while avoiding a group that no longer
can minister to its members. There is no provision for charismatic
leaders who desire to have a large small group, they must divide,
and train new leaders to be put in place. The church is \textbf{never}
a one man show, unless that man be Christ dying on the Cross.


\subsection{Teaching}

Depending on the differing gifts enabled, so must the gifts be used.
If there be more then one teacher, then they must teach. In Crust
Groups, there must be a rotation of speakers, teaching topically or
energetically as needed. (1 Corinthians 14:26-33, Ephesians 4:1-16,
Ecclesiastes 4:13-16, Proverbs 24:6). In the crust groups, teachers
teach from their seat (located as part of the great circle), and while
he speaks, others listen.


\subsection{Bible Reading}

At the start of each and any meeting, a passage of Scripture is read
aloud. Core Groups systematically work their way through different
books of the Bible, so that no portion of Scripture is neglected.


\section{Definitions}


\subsection{Trinity}

God the Father, God the Son, and God the Holy Spirit are 3-in-one,
yet each part is distinct.


\subsection{God}

The almighty Father


\subsection{Jesus}

Jesus is our only way to God, and we cannot on our own merit gain
entrance to the King of Kings without first accepting Jesus as the
Son of God, and making him Lord and Savior of our Life.


\subsection{Holy Spirit}

The Holy Spirit indwells every believer who confesses that Jesus Christ
is Lord, and through the Spirit God grants different gifts to the
members of his body, so that all might be built up.


\subsection{Bible}

The Bible is the basis for truth, and anything added to it or taken
away from it weakens the body of Christ. Add preaching must be from
the Bible.


\subsection{Sin}

Man is inherently sinful and doomed to destruction. It is the grace
of God that saves man. Without believing in Christ, we cannot be saved.
If is not by our own righteousness that we can be saved. No amount
of rules and regulations can make us right with God, for we are all
sinners at heart. A true belief in Christ entails repentance and and
change of heart.


\subsection{Church}

The church is a group of believers who have been called out of the
world to meet together. A church is defined by the following 4 characteristics:
\textbf{Believers} who meet \textbf{regularly} together for the \textbf{purpose}
of spiritual growth \textbf{belonging} to a permanent location. The
people of an church continue to be an church even when they are not
formally meeting together. In all circumstances, it is ideal for believers
living in the same general location to consider themselves once church
\end{document}
