%% LyX 2.0.6 created this file.  For more info, see http://www.lyx.org/.
%% Do not edit unless you really know what you are doing.
\documentclass[english]{article}
\usepackage[T1]{fontenc}
\usepackage[latin9]{inputenc}
\usepackage{babel}
\begin{document}

\title{Home Constitution}


\author{David Guenther}

\maketitle

\section{Purpose}

Planet Home's focus is on establishing healthy, thriving families.
This is established by the physical support, mental support, and spiritual
support that parents offer to their children through relationship.


\section{Roles}


\subsection{Mothers}

Mothers provide the physical, emotionally, and genetic support that
a child needs at their earliest stages. A child cannot function without
a mother for the first nine months of their life. It is in the best
interest of the child therefore, the maintain this connection for
the rest of their life. In the case of a dysfunctional family: Planet
Home seeks rather to help the original mother of the child find help
in raising a child, rather then to remove the child from the situation.
However, if the mother is unwilling to change, measures might need
to be taken to provide a safe home for the child.


\subsection{Fathers}

As a child also cannot be created without the help of a father, and
genetic code is passed from father to child, it is imperative that
a child know their father. It is also crucial that a father spend
time with their child, and show their love to them. Love of a father
prevents many emotional problems later in life.


\subsection{Younger Children (>=12)}

Younger children cannot care for themselves fully, and need supervision.
Food, Water, and Sleep are not enough for these children to thrive.
They need to have security, fun, and love in addition to grow.


\subsection{Older Children (<12)}

Middle children represent a key transformation period. They go from
playing with toys, to playing with real life. No longer are their
amusements purely theoretical, but they transfer into situations that
have significant consequences. It is at this point that the child
is given more and more privileges to aid in their maturity development.
In addition, they might be required to care for younger children as
well, or work to help provide for the families income.


\section{Layout}


\subsection{Co-Ed Family}

A family should not consist of more then 12 individuals, with a mother
figure and a father figure. It is ideal {[}if not mandatory{]}, that
this couple be married. In planet Home a family name is given to those
who belong to a family. It is recommended children keep their original
last name, however, substituting it as a middle name. This family
is for life, and the kids will be legally adopted as children of the
new parents. It is not recommended that these parents take more then
12 children during their time as parents, as these connections should
be built for life. The more connections, the less time can be given
to each individual. This base unit of a family is designed to eat
together, often play together, and live alongside each other. A Family
should consist of both boys and girls, however, extreme care must
be given. If children are raised alongside each other from a young
age, there will be significantly less sexual attraction. Well functioning
family should be the safest place to be, for there must be much love.


\subsection{Single Gender Family}

A separate contingent of family groups will be established for the
development of kids who enter into Planet Home at a later age (10
and up). These children still need all the love and support that the
Co-Ed Family's provide, and so the structure will be identical, with
the exception that this family has only one gender.


\subsubsection{Mentorship Family}

It may come that a boy or a girl comes to planet Home needing help,
however, as they are already over the age of 16, and feel a strong
need for independence, forcing a family upon them might have a detrimental
effect. What they need is help, but not things handed to them, for
they are already at a place in life where they can provide for themselves.
A mentorship family provides the love and support necessary to build
these children up, without forcing a family they do not know on them.
Could be run by a single person (same gender as mentoree)


\section{Functions}


\subsection{Marriage}

Marriage marks the beginning of a new family, and should not be discouraged
for the older children. The new couple then will have to leave Planet
Home, and start a new family somewhere else (hopefully reasonable
living quarters can be provided for). This breaks the cycle of physical
poverty given to these children, and allows them with a solid foundation
to start a new life. There is no age limit on marriage, but rather
a focus on maturity of the individuals to spend their life together.
The host parents need to be involved in the establishment of this
new marriage, providing counsel for the newly-weds. All those wishing
to get married must have the firm mindset that marriage is forever.


\subsection{Family}

It is important to have homes in the context of family. Individual
homes should make an effort to meet with family members often. For
those who do not have family, homes will be paired with homes for
the purpose of family.


\subsection{Friends}

Children also need to have friends, and having families interact with
each other brings about an opportunity for kids to meet and play with
other kids.


\section{Integration}


\subsection{Church}

Students should be provided with spiritual instruction, however, it
should not be the role of those with the spiritual gift of teaching
to teach those students. Families have the obligation to teach their
children much of what the Bible says, from praying before meals, to
doing devotions at night, but a child also needs to be with other
believers who commonly meet together. It is this environment that
provides the growth needed to transform a new Christian into a strong
Christian. A family must make the attempt to continue to meet with
other believers regularly for the purpose of spiritual growth. This
is accomplished by the core group in church. A single family meets
with other families to compose this group. Meeting regularly, they
encourage each other in life.


\subsection{School}

School is a must for all kids, and all students go to school.

Children should be taught by their parents or family member until
the age of 11. If this is an impossibility, the school does provide
education at the facility, but it is with involvement from parents
attending a church.
\end{document}
